\RequirePackage[l2tabu, orthodox]{nag}
\documentclass[a4j,oneside]{jsbook}
\usepackage[dvipdfmx]{graphicx}
\usepackage{pxrubrica}
\def\――{―\kern-.5zw―\kern-.5zw―}
\makeatletter
\def\WordCount#1{%
  \@tempcnta\z@%
  \@tfor \@tempa:=#1\do{\advance\@tempcnta\@ne}%
  {#1}%
  (\the\@tempcnta 文字)
}
\makeatother
\title{ロックの人格同一性論\\ \――意識・自己・人格\――}
\author{学生番号04-190108 小林大晃}
\date{\today}
\addtolength{\fullwidth}{-26truemm} %全体の幅(ヘッダ部の幅)を既定値から26mm小さくする
\setlength{\textwidth}{\fullwidth}  %本文の幅(textwidth)を全体の幅(=ヘッダ部の幅)にそろえる
\setlength{\evensidemargin}{10truemm}   %偶数ページの左余白を10mm(+1インチ)にする
\setlength{\oddsidemargin}{10truemm}    %奇数ページの左余白を10mm(+1インチ)にする

\begin{document}
\maketitle
\tableofcontents
\begin{flushright} 
以上、39, 512字
\end{flushright}
\newpage
\chapter*{
凡例
}\addcontentsline{toc}{chapter}{凡例}
\begin{enumerate}
\item『人間知性論』(原題:{\itshape An Essay concerning Human Understanding})については、Nidditch版を底本とする。
\item『人間知性論』の参照については、Nidditchの章分けに従って(巻.章.節)の記号で表す。
\begin{itemize}
\item[例](2.27.1)は2巻27章1節を指す。
\end{itemize}
\item 本文中で引用したテクストの日本語訳は断りのないかぎり論文著者による。訳出に関しては大槻訳と先行する邦語論文を参考にした。訳者による補足を〔亀甲括弧〕で、省略を〔…〕で表す。原文の省略はピリオド三つ … で表す。
\item 底本でイタリック体で強調されている箇所には訳出の際に傍点を付している。本論文著者による強調の場合は、傍点を付した箇所の直後に〔強調は小林〕と明記しておく。
\end{enumerate}
\chapter*{
序論
}\addcontentsline{toc}{chapter}{序論}
人格の同一性、あるいは人格同一性(personal identity)とは、一般的にいえば、われわれの同一性である。この人格の同一性は、よくよく考えてみると、案外あたりまえな出来事ではない。あくまで想定だが、もし、あなたが今完全に記憶を失ってしまったら、あるいは、もしあなたが、身体と記憶状態をある場所でスキャンしたのち身体を一瞬で分解してしまって、同じ構造の身体と同じ記憶状態を別の場所で再構成するような瞬間移動装置をくぐってしまったら、あなたはその前後で同じ人格なのだろうか。このように考えると、今のあなたの人格と、過去のある時点であなたであった(かもしれない)人格が同じであるとはどういうことかは、哲学的に問われるような問題であると十分納得できるだろう。
\par
人格同一性について考えることには単に思弁的な面白さだけではなく、実践的な重要性もある。なぜなら、法律による処罰の正当性を担保するのが、人格同一性だからである。罰は責められるべき行為をしたまさにその行為主体に課せられるべきであり、そうでなければもはや罰でなく単なる暴力である。人格概念はここでいう罰の対象となる主体であり、それが罪をなした時点と罰を受ける時点で同じであることが、処罰には必要なのである。
\par
さて、人格同一性をめぐる問題を見つけ、議論の先鞭をつけたのは、ジョン・ロックである。ロックはその認識論における主著『人間知性論』\footnote{以下、本論文では本書を『知性論』と表記することがある。}の第2巻27章「同一性と差異性について」\footnote{以下、本論文では本章を指して、「同一性章」と表記することがある。}で、はじめてこの問題を明確な形で定式化し、同時代や後世の哲学者に多大な影響を与えた\footnote{Noonan(2003), p. 24.}。しかし、『知性論』は300年以上も前の著作であるにもかかわらず、現在でもなお彼の人格同一性論の解釈、ならびに成否の判定をめぐってさまざまな説が提出されており、いまだ統一的見解は得られていない。特に問題になっているのは、ロックが「同じ意識(the same consciousness)」という表現で何を表しているかである。ロックは同一性章を通じて繰り返し、「同じ意識が同じ人格をつくる」\footnote{同一性章16節の欄外要約}と述べている。しかし、ロックは意識がいかにして同じになるのかについては明言していない。そのため、この「同じ意識」という表現の解釈が彼の人格同一性論の解釈をめぐる中心的問題のひとつである。
\par
管見の限り、ここには次の三通りの解釈の方向性がある。ひとつめは記憶説、すなわち、以前の思考・行動を記憶していることが、その時と同じ意識を持っているということだとする解釈で、古くはトマス・リードから、最近ではディッカー\footnote{Dicker(2019)}まで多くの論者がこの解釈を支持している。ふたつめが専有説という解釈である\footnote{専有説という分類名は、今村(2010)に負っている。}。以前の思考・行為を自分のものとして気にかけていることが、その時と同じ意識を持っているということだとする解釈で、ビーハン\footnote{Behan(1979)}やウィンクラー\footnote{Winkler(1981)}がこの立場をとる。そしてみっつめが三人称的構成説、すなわち、以前ある行為をしたときと同じ意識を持っているということは、第三者によって判断されることだとする解釈である。この立場は一ノ瀬\footnote{一ノ瀬(1997)}がとるものだ。これらの三つの解釈の方向性には、それぞれロックの人格同一性論の解釈にあたって有効な点と問題になる点とがあり、依然として解釈の余地が残されている。
\par
今挙げた三つの解釈のあいだには様々な違いがあるが、特に大きな対立は、意識の同一性ひいては人格の同一性を一人称的に構成されるものとして捉える立場と(これは記憶説と専有説が共通してとる立場である)、それを三人称的に構成されるものとして捉える立場の対立である。この対立は簡単には解消することができない。なぜなら、意識の同一性が一人称的に構成されるとする議論(これを今後、意識同一性の一人称的構成説と呼ぶ)はロックのテクストを素直に読むことで得られる反面、人格同一性に必要とされる基準(standard)を満たすことができないように見える一方で、人格同一性が三人称的に構成されるとする議論(これを今後、意識同一性の三人称的構成説と呼ぶ)はこの基準を満たすことができるけれども、反面、ロックのテクストと折り合いが悪いように思われるからだ。
\par
これらの背景から、本論文は、この一人称的構成説と三人称的構成説を仲介し、ロックの人格同一性論をテクストにより即した形で、しかもそこで得られる人格同一性の原理が人格同一性が満たすべき基準をよりよく満たしうるような形で解釈することを目的とする。一人称的構成説と三人称的構成説との橋渡しの鍵となるのは、ロックが同一性章で述べる、「\kenten{人格}は〔…〕この\kenten{自己}に対する名(name)である」(2.27.26)という命題である。この命題は、人格概念をめぐって、一人称的な側面に深く関わる自己(self)と、第三者とのコミュニケーション手段とされる名\footnote{『知性論』第3巻で、ことば(Words)は第三者とのコミュニケーション手段としての位置付けを与えられている。}が結びつくことを予感させる。それにもかかわらず、管見の限り、この命題をそれ自体として十分に解釈しようとする試みは見られない。そこで本論文は第一に、この命題に解釈を与えることを目標とし、第二に、その際に用いる解釈枠組みによって、先ほどの一人称的構成説と三人称的構成説との調停を図ることとしたい。
\par
本論文の構成を以下に述べる。第1章では人格同一性論に入る前に、ロックの同一性論の全体に通底する理路を確認する。第2章では、ロックの人格同一性論の、特に意識の同一性をめぐる三通りの解釈をそれぞれ検討し、意識同一性の一人称的構成説(つまり記憶説と専有説)と三人称的構成説の長所と短所を検討する。最後になる第3章では「人格は、この自己に対する名である」という命題の解釈から、一人称的構成説と三人称的構成説を調停する解釈枠組みを提示したい。
\chapter{
ロックの同一性論の基本的な理路
}
同一性章では人格のほかに原子、原子のかたまり(mass)、動植物、そして人間の同一性も考察されている。同一章ではまた、これら特定の種類のものの同一性だけではなく、同一性という関係の観念一般をめぐっても考察がなされる。ロックはこれら同一性一般の検討や、人格以外の種類の事物の同一性の考察、また事物の種(species)と同一性の関係についての探究を下敷きにした上で、同一性章の中盤から人格同一性を考察し始める。よって、彼の人格同一性論の検討に入る前に、まず彼の予備的な議論を確認しておく必要があるだろう。本章では、ロックの同一性論の理路をテクストの流れに沿って確認する。
\section{
同一性の二つの原理
}
同一性章を最初から辿っていく。ロックが同一性章1節で最初に語っているのは、同一性という関係の観念を人間が発見する機会(occasion)である。関係の観念は心が持つ既存のある観念を、もうひとつの他の観念と比較することで得られる観念だったが(2.12.7)、中でも同一性という観念は、「ある事物がある確定した時点と地点に存在すると考える時、〔…〕それをもうひとつの時点に存在するそれ自身と比較して」(2.27.1)形成されるものである。ここで、ロックの問題とする同一性の観念は基本的には異なる時点に存在する事物どうしの同一性、すなわち通時的同一性である。というのも、ある事物の共時的同一性は誰もがただちに確認することだとロックが述べるように(2.27.1)、ロックにとって事物の共時的同一性は前提だったからである。
\par
これに続いて同一性という関係が成立する仕方が語られる。同一性が成立する際に前提される最初の命題は、「ある地点にある時点で存在する事物はすべて、同種のあらゆるものを排除し、そこにはそれ自身のみが存在する」(2.27.1)ことである。ヤッフェはこれを「いかなる同種の二つの事物も、同じ場所を同じ時点で占めることはできない」と言い換え、この命題を「地点-時点-種類原理(Place-Time-Kind Principle)」と呼ぶ\footnote{Yaffe(2007), p. 196.}。これがロックの同一性の原理のひとつめである。
\par
ここでなぜ事物の種類が問題になるのかというと、異なる種類の事物であれば、同じ特定の時点と地点に共存することが可能だからである。ロックは、実体を神、有限な知的存在者、そして物体(bodies)の三つの種に分類し、異なる二つの実体でも、その種が異なれば同じ時点で同じ地点に存在することができると述べる。ロックにおいて同一性の原理はひとつのものが他のものを排除するところに存していた。よって、同一性が問われる二つの事物それぞれがどの種に属すかが問題になったのである。
\par
この原理からもうひとつの命題が帰結するとロックは考えている。それは「ひとつの起源(beginning)をもつものは同じ事物であり、それとは時点と地点において異なる起源をもつものは同じでなく異なる」(2.27.1)というものである。この命題を本論文では「単一起源の原理」と呼ぶこととしよう。これが同一性の原理のふたつめである。ここに至るロックの議論の流れを整理すると、次のようになる。まず、地点-時点-種類原理が前提される。すると、ある事物は任意の時点で他の事物\kenten{ではない}ものとなる。そして、ある事物の同一性を問う際、その事物は過去のある時点・地点で他のどんな事物でもなく、それ自らが共時的に同一だったものと関係する(refers to)。そこから、ひとつの事物が二つの起源から生じることはなく、二つの事物がひとつの起源から生じることもないということが帰結する。そしてこの帰結から、単一起源の原理が導かれることになる。
\par
ここで、地点-時点-種類原理と、単一起源の原理では、同一性という関係に対して寄与する仕方が異なることを指摘しておきたい。地点-時点-種類原理は、同一性という関係が成立するための必要条件である。これがなくては、「同一性と差異性という思念あるいは名は無駄」(2.27.2)なのだとロックはいう。これに対して、単一起源の原理は、なにかとなにかが\kenten{実際に}同じであるための条件である。これは同一性が成立するための十分条件ということができる。この区別が、今後特定の種の事物の同一性について考える際、考察のステップを二つに分けることになるだろう。
\par
さて、ここまでに挙げた二つの原理が、ロックのいう同一性の原理である。ここで、地点-時点-種類原理についてライプニッツが『人間知性新論』で反論している。彼は、二つの光線は同種のものであるにもかかわらずお互いを排除しあわない\footnote{Leibniz(1765), p. 230.(2巻27章1節)}という例を出して、このような場合、地点-時点-種類原理は成り立たないにも関わらず、光線どうしはその通路というほかの手段によって区別されることを示した。ライプニッツは、ロックの地点-時点-種類原理が同一性の必要条件のすべてではないことを指摘したのである。
\par
ライプニッツの指摘するように、確かにロックは区別の内的原理という考えには至らず、事物にとって外的な時間と場所が外的にそれらを区別すると考えた。しかし、区別にいたる仕方に難点があれど、ロックも\kenten{区別}(distinction)について語っていることには間違いがない(2.27.2)。ロックはその区別の\kenten{仕方}については不十分な描き方をしてしまったが、その視線の先はおそらくライプニッツと同様に、個体(individual)であることが同一性には必要だという方向を向いていたといえるはずだ。ここで個物あるいは個体とは、さしあたり、ひとつのものとしての統一性をもち、みずから以外の何物とも区別されていて、数的に同一である事物のことを指す。ロックの同一性に関する議論はすべて、このひとつの個物としての事物の同一性の議論だということができる。
\section{
個体化の原理
}
ロックが同一性と個体であることの緊密な繋がりを見てとっていたロックは、次に個体化の原理(principium individuationis)に言及する。ロックは同一性に関する先ほどの考察を受け、個体化の原理を「任意の種(sort)のある存在者を、同じ種類(kind)の二つ〔以上〕の存在者たちに伝達できないある特定の時間と場所に限定する、現実存在それ自身」(2.27.3)と特徴づける。
\par
ここで個体化の原理が探究されるのにはもうひとつの、次のような理由がある。まず、本節以降では、これまでの同一性一般についての考察から特定の事物における同一性の考究へと主題が移行する。特定の事物における同一性判断の際には、問題となる事物の種に固有の個体化の原理を参照することが必要になってくることがその理由だ。実際にロックの同一性の議論を検討することによって、個体化の原理が事物の同一性の説明でどのような役割をもっているか見てみよう。
\par
彼がまず考察対象とするのは原子である。そして原子にかんしていえば、個体化の原理はその現実存在そのものである。なぜなら原子は「ひとつの不変な面に覆われた連続したある物体」(2.27.3)で、\kenten{固性}(solidity)をもつものだからである(2.4.1)。この固性とは、不可入性(impenetrability)とも呼ばれるもので、二つの異なる物体が互いにどれだけ近づいてもにお互いの内部に侵入しないという性質のことである(2.4.1)。こうした特徴をもつひとつの原子は、その表面において、固性によってつねに他の原子とは区別されたものである。よって、ある特定の時点と地点に現実存在するかぎり、そのこと自体によって、ある原子は他のいかなる原子もその地点・時点には存在できない自分の地点・時点に存在する個体なのである。
\par
このように原子の個体化の原理を説明したのちに、ロックは原子の同一性を説明しはじめる。ロックいわく、こうした原子の\kenten{ひと粒}は、「その〔現実存在が連続する〕限り、それだけ同じであり、他のいかなるものでもない」(2.27.3)。ここまでが、同一性章における原子の同一性の説明のすべてである。
\par
ここでの原子の同一性の説明は二ステップからなっている。最初のステップが、原子の個体化の原理の解明で、次のステップが、個体である原子の現実存在が連続することの指摘である。ここで次のように考えることができそうである。すなわち、この各ステップでは、先ほど確認した二つの原理を満たすことが、すなわち前者では場所-時間-種類原理を、そして後者では同一起源の原理を、原子という種のひとつの事物が満たすかどうかが確認されているというふうにである。しかし、そのためには、個体であることが場所-時間-種類原理を満たすことに対応することと、連続的に現実存在することが同一の起源をもつことに対応することの両方を説明しなくてはならないだろう。前者がいえる理由を説明する。本章1節の最後で述べたことに戻るが、おそらくロックは他のものから区別された個体であることが同一性に必要であり、その区別の原理が地点-時点-種類原理にあると考えていた。この考え方まで遡れば、個体であることを示してしまいさえすれば、ロックにおいては論理的に地点-時点-種類原理もすでに満たしていることになることは明らかである。つづいて後者の理由を説明する。まず、連続しているということばで、ある時点から別のある時点まで途切れずにその状態であるということを意味しているとする。すると、ひとつの起源からずっと個物として他のものとは区別されたまま連続して存在する事物は、その起源が他の個物の起源と混ざることがない。よって、個物は個物として連続的に現実存在している限り、ずっと最初の一点に起源を求められるのである。ロックがこうした考え方をしていた傍証として、彼は「連続する現実存在が同一性をつくる」(2.27.29欄外要約)という言い方をすることがある。このように、連続的に存在していることは単一起源であることと同様の内実をもつ同一性の原理である(今後、これを「連続性の原理」と呼ぶことにする)。こうして、個体化の原理の解明から連続性の確認というステップは実際に、場所-時間-種類原理と(連続性の原理を介して)単一起源の原理を満たしていることを確認するステップなのである。
\par
果たして、この二ステップの説明は原子の同一性についてだけあてはまるのか、それとも、原子以外の種類の事物の同一性にもあてはまるのか。もし人格同一性の議論にもこれが当てはまるのであれば、人格同一性論を解釈するための強力な枠組みを得たことになるだろう。次節でこれを検証していきたい。
\section{
各種事物の同一性の説明
}
\subsection*{
物体の場合
}
物体の同一性の説明は、ロックがこれを「〔原子の場合と〕おなじように」(2.27.3)と書き始めていることからもわかるように、原子のときに用いた二ステップの説明がそのまま使われている。まず、物体の個体化の原理は、まずあるひとつの物体に\kenten{部分として}結合している原子のひとつひとつが先ほどの仕方で同一であること、かつ\kenten{全体として}現在の原子構成からいかなる追加、脱落、交換もないままであること、これらの両方を満たした構成の現実存在である。これに関しては実際、この構成がある時点である場所を占めるかぎり、他のいかなる物体も同じ時点で同じ場所に存在することはできない。$というのも、仮にひとつの物体Aが占めるのと同じ時点と場所にもうひとつの異なる物体Bがあったとして、その物体Aと物体Bはそのとき合体して、ひとつの、AともBとも異なる物体になってしまっている状態でしか考えられないからである。しかるに、この合体した物体はその合体の時にそれ以前にあった物体とは構成が変わって新たに生まれた事物であり、それ以前に存在していた物体A、物体Bのいずれとも起源が異なるので、先に示した単一起源の原理、すなわち、同じものはすべてひとつの起源をもつという原理より、物体Aとも物体Bとも同じではない$。こうして、件の構成は実際に個体化の原理であるとみなせる。そして、個体化の原理たる構成が「結合して現実存在しているあいだ(while)」(2.27.3)、つまり現実存在が連続している間、ひとつの物体は同一なのである。
\subsection*{
植物の場合
}
ついで、ロックは植物の同一性について次のように述べる。ロックは、生命ある身体(living Body)の場合、「\kenten{同一性}が〔物質の場合と〕同じものに適用されるのではない」(2.27.3)と述べてから、続く4節で次のように述べる。
\begin{quote}
すると、ひとつの共通の生命をともにする(partaking of)ひとつのまとまった身体(Body)のうちに諸部分のこうした組織秩序(Organization)をもつものがひとつの植物なのであり、それが同じ生命をともにするかぎり〔…〕同じ植物であり続ける。それというのも、あるひとつの瞬間で、あるひとつの物質の集積のうちにあるこの組織秩序は、その特定の凝集体のうちでほかのいかなるもの〔特定の凝集体〕からも区別され、そしてその個体なる生命なのである。それ〔個体なる生命〕はその植物の生きている身体に合一された、感知されずに代替わりする(succeeding)諸部分の同じ連続性のうちで、その瞬間から前後両方にむかって常に現実存在する。そしてそのゆえに、それ〔組織秩序〕は同一の(same)植物をつくり、それ〔植物〕の全ての部分を同一の(same)植物の部分にするその同一性をもつのであり、〔これが続くのは、〕それら〔植物の諸部分〕があの連続する組織秩序\――これはじぶんに合一されたすべての部分にあの共通の生命を運ぶのに適している\――に合一されて現実存在しているすべての時間のあいだ〔である〕。(2.27.4)
\end{quote}
ここでロックは個体化の原理という言葉を直接は用いない。しかし、組織秩序(organization)があるひとつの植物をほかの植物から\kenten{区別し}、\kenten{個体なる生命}とするという内容からは、この組織秩序の現実存在が植物における個体化の原理であると読むことができるだろう。ここで組織秩序とは、例えば、オークの葉の葉緑体が光合成によって養分を作り出し、それが師管という構造体を流れて種子に流れ込み、種子が自己複製を達成する、などといった、植物のそれぞれの部分が組織として固有の役割をもち、それらの役割どうしがひとつの秩序ある体制をなしている状態のことである。そして、この組織秩序がある時点で、あるもののうちで現に成り立っているとき、この、そのうちで組織秩序が成り立っているもの(すなわち身体)の占める場所を組織秩序が占める場所だということにすれば、他の組織秩序は同時に同じ地点に現実存在することができない。なぜなら、ある特定の時点である場所を占める身体は、さきほど物体がそうであったように、ほかのいかなる身体もその占める場所から排除するからだ。こうして、任意の各時点において異なる身体が必ず互いに区別されることから、その身体のうちで成立する組織秩序もまた必ず互いに区別される。よって、組織秩序が現に成立していることは、個体化の原理の要件を満たすのである。
\par
そして植物の同一性は「〔植物の諸部分〕があの連続する組織秩序に合一されて現実存在しているすべての時間のあいだ」(2.27.4)保たれるのである。よって、植物の同一性に関しても、最初に個体化の確認、次に連続性の確認というロックの同一性論に通底する議論の流れに沿っているということができるだろう。
\subsection*{
動物、人間(Man)の場合
}
そして、「ことは\kenten{動物}でもそうたいして違わない(The Case is not so much different in {\itshape Brutes})」(2.27.5)。動物の場合も、個体化の原理はこの組織秩序であり、それが連続的に成り立っているあいだ、ひとつの動物の事物は同じ動物でありつづける。そして人間(Man)そして人間の同一性も同じように成立するのである。なぜなら、ロックは「人間」(Man)という語で動物としての人間のことを表すからである(2.27.8)\footnote{のちに述べるが、ロックは人間(Man)と人格(Person)の概念を区別する。}。
\par
ここまでの物体、植物、動物、人間の同一性を鑑みるに、前節の最後で立てた問い、すなわちロックは事物の個体化の原理を解明してから、連続性をもつことを確認して同一性の説明としているかどうかについては、実際にそのようにしていると考えてよさそうである。すると、同じ章でこれから問われる問題、人格の同一性の問題であっても、ロックはさしあたりこのステップを頼りに考えているのではないかと仮説を立ててもよいのではないだろうか。
\par
次章ではこのことを踏まえてロックの人格同一性論をめぐる諸解釈の比較検討に望みたい。しかしその前にひとつだけ、ロックの同一性論の重要な特色を紹介、検討したい。その特色とは、ロックの同一性論が、いわゆる相対的同一性説(relative identity theory)とでもいうべきものだということである。
\section{
相対的同一性説
}
相対的同一性説とは、ある事物の同一性の成立は、その事物が何の種に属するものとして考えられているかに関わる(relative to)とする考え方である。この相対的同一性説の定式化にはいろいろあり、それがロック解釈として適当か否かをめぐっては意見が分かれている\footnote{たとえばLowe(2005)などが賛成し、Alston \& Benett(1988)やYaffe(2007)はこれに反対している。}が、少なくともロックが実際にそういった考え方をしていたことはテクストのうちにはっきり見て取ることができる。ロックは同一性章27節で次のように述べる。
\begin{quote}
それゆえ、実体の合一(Unity)がすべての種類の\kenten{同一性}を包括するわけではないし、すべての場合でそれ〔同一性〕を決定するのでもないだろう。しかし、それ〔同一性〕を正しく考え判断するには、それ〔同一性〕が当てはめられることばが何の\kenten{観念}を表すかを考えなくてはいけない。もし\kenten{人格}、\kenten{人間}、\kenten{実体}が三つの異なる\kenten{観念}を表す三つの名(Names)であるなら、同じ\kenten{実体}であることと、同じ\kenten{人間}であることと、同じ\kenten{人格}であることとはそれぞれ別のことである。というのは、その名に属する\kenten{観念}であるようなもの、そういったものこそが\kenten{同一性}であるにちがいないからだ。(2.27.7)
\end{quote}
\par
引用文中の第二文目が、まさに相対的同一性説の考え方である。当該文の「ことば」は次に「名」と言い換えられているが、ロックが「ある種のものであることと、その種の名への権利をもつこととは、同じこと」(3.3.12)と述べていることを考慮すると、同一であることと一口に言っても、種ごとにその内実は異なるとロックが考えていたことは明白である。ほかにも、引用部分の最後でロックは、同一性という観念は「名に属する\kenten{観念}であるようなもの、そういったもの〔…〕であるにちがいない」と述べる。名に属するというのは、同一性は必ず「\kenten{これこれ}(ある名、例えば原子、動物など)\kenten{の}同一性」という仕方で語られるものだということだと考えられる。これらからわかるように、同一性を種と密接に関係させる考え方をロックは持っていたことは間違いなさそうである。つまり、同一性の探究の際には、先ほど述べた二ステップ以前に、同一性が論じられる事物の種がなにで、その種は一体どういう本質をもつものかが問われなければならないことになるだろう。
\par
ここまでで、ロックの人格同一性論を解釈するための土台を作ってきた。まとめると、すくなくとも同一性章7節まで、ロックはあるひとつの事物の通時的同一性を考察する際、次のような基本的な理路で考えているはずだと考えられる。まず、対象の事物がどの種に属するか(どの名を与えられるか)を考え、その種の観念がどのようなものかを考える。そして、その種の事物の個体化の原理を考える。これによって個体化されているあいだ、対象の事物は場所-時間-種類原理を満たす。そうすれば、その対象が連続的に現実存在をしているあいだは、その事物はそのあいだずっと単一のはじまりを持っており、そして同じものでありつづけるのだ。さしあたりこれにロックがのっとっていると仮定して、次章で人格同一性論を検討、解釈する際の指針としていきたい。
\chapter{
ロックの人格同一性論の諸解釈とその問題
}
本章では、最初に前章で確認したロックの同一性論の基本的な理路に沿ってロックの人格同一性論を確認していく。続いて、ロックの同一性論を理解するにはロックのいう「同じ意識」がなにかを明確にしなければならないことを示し、この「同じ意識」について論者たちが提出してきた諸解釈を記憶説、専有説、三人称的構成説の三つに大別し、それらの長所、短所の検討をする。そして最後に、同じ意識とロックがいうものが一人称的に構成されるものとする立場と、三人称的に構成するものとする二つの立場があり、両者のあいだには、解釈の変更だけでは調停しきれない断絶があるという問題を示す。
\section{
人格の定義と意識による個体化
}
前章に示した人格同一性論の基本的な理路に従えば、人格の同一性を考えるにあたってまず、「人格」という名が結びつけられる観念(すなわち種)が何であるかを検討しなくてはいけない。そしてロックはすぐさまこれに取り組む。ロックは、同一性章9節で人格を次のように定義する。
\begin{quote}
〔…〕〔人格が表すのは、〕思うに、理性と内省をもち、それ自身をそれ自身だと、〔つまり〕異なる時間と場所で同じ思考する事物(thinking thing)だと考えることができる思考する知性的存在者(Being)である。(2.27.9)
\end{quote}
ここでの「思考する事物」(thinking thing)という言葉遣いについて多少補足したい。大槻は、ロックにおいて‘thing’がなかば専門語的に使われていると指摘する。大槻によれば、「いっさいを対象化して静止的にとらえるかれでは、事物〔ここでは物質的な事物の意味か〕だけでなく、事象も思惟も『事物』の意味の thing とされ、精神も同じことばで表される」\footnote{ロック(1690), 大槻訳, 第一巻(1972), p. 244.}。ここでロックが思考する事物と述べる場合、思考する物体のかたまりというだけではなく、精神(Spirit)あるいはたましい(Soul)まで射程に含んでいると考えられるだろう。
\par
こうして人格の概念が劃定された。次に問題になるのは、この人格という種に固有の個体化の原理が何であるかだ。ロックは直接には個体化の原理という言葉を用いないが、代わりに次のように述べる。
\begin{quote}
意識(consciousness)は常に思考に同伴しており、そしてそれ〔意識〕が、全ての人を彼が\kenten{自己}({\itshape self})と呼ぶものにするものである。そして、このことによって(thereby)、〔意識は〕彼自身を他のあらゆる思考する事物から区別する(distinguish from all other thinking things)。ここにのみ、\kenten{人格同一性}、すなわち理性的存在者の同じさ(sameness)は存する。(2.27.9)
\end{quote}
意識は、人を自己と呼ぶものにすることによって、「彼自身を他のあらゆる思考する事物から区別する」。事物を他の同種の事物から区別するのが、個体化の原理である。人格同一性について考える際も、この意識が個体化の原理であると考えてよさそうである。
\par
こうして、意識が人格の個体化の原理であることがわかったが、意識作用が\kenten{どのように}人格である事物を個体化することができるのかも検討しなくてはならないだろう。そのためには、そもそも意識がどういう作用かを考えなくてはならない。手がかりになるのは、先ほどの人格の定義につづく次の叙述だ。
\begin{quote}
〔…〕〔人格〕はそれ〔それ自身をそれ自身だと考えること〕を、思考から切り離すことのできないあの意識のみによって行う。そして意識は、私の見るところ、思考にとって本質的(essential)である。だれもが、自分こそがまさに知覚しているということを知覚せずには知覚することができない(It being impossible for any one to perceive, without perceiving, that he does perceive)からである。何かを見るとき、聞くとき、味わうとき、感じるとき、熟考するとき、あるいは意志するとき、われわれは自分がそうしていることを知っている。このように、それ〔意識〕とは、つねにわれわれの現在の感覚や知覚に関するものなのであり、これによってすべての人は彼〔その人〕自身にとって、彼〔その人〕が\kenten{自己}({\itshape self})と呼ぶものなのである〔…〕。(2.27.9)
\end{quote}
ここでロックのいう意識は、ビーハンのいうような「思考の再帰的知覚」\footnote{Behan(1979), p. 64.}の作用だろう。視覚にしろ、味覚にしろ、あるいは意識にしろ、ある対象を知覚する際には、自分がそれを知覚していることを知覚してしまう。つまり、$ある対象xを知覚することをP(x)$とするならば、$P(x)→P(P(x))$も成立するのである。そして、この二段目の知覚のほうを、ロックは意識と名付けているのである。
\par
この意識概念には、もうひとつ別の側面がある。ヤッフェはそれを、意識が「世界のある部分を自分自身の部分だと認識する、あるいは『考える』精神的作用」\footnote{Yaffe(2007), p. 214.}であるという。ロックはたとえば、「この意識のもとに包括されているあいだ、小指が\kenten{自己}の部分であることは、自己の最大の部分であるものに劣らない」(2.27.17)と述べる。意識によって自己に包摂されているものはそのことによって、他の意識に包摂されているものとは区別される。そして、このことによって、意識は思考する事物を個体化することができるのである。
\par
そして、この個体化の原理である意識が連続するあいだ、ひとつの人格は同じ人格である。ロウは、ロックは動物の同一性と人格の同一性の類比によってそのことを説明していると指摘する\footnote{Lowe(2005), p. 214.}。
\begin{quote}
異なる身体(Body)が、同じ生命によって、ひとつの動物\――その同一性はその実体の変化のうちで、ひとつの連続する生命の統一性(Unity)によって保たれる\――に合一される(united into)のと同様に、異なる実体は、同じ意識によって(実体がまさにそれ〔意識〕に参与する点で)ひとつの人格に合一される。(2.27.10)
\end{quote}
ここでは、身体と実体、生命と意識、動物と人格という語がそれぞれ対応している。ロックは意識が人格の同一性に関してもつ役割を、動物の同一性における生命のもつ役割と類比的なものだと考えているのだ。同一性章3節で、生命は「というのもこの組織秩序は〔…〕その個体なる生命である」(2.27.3)と、組織秩序とほぼ同義の語として使われていた。そして、人格の同一性においてはこの組織秩序とそれが作る生命の立ち位置を、意識が占めることになる。ある人格を構成する実体は、例えば指を構成する分子は刻一刻と入れ替わるけれども、それがその人の身体にあるあいだ、その分子は指ごと自分の一部として意識され、気にかけられる。この意識が現に作用しているあいだ、人格を構成するものがどのように変化しても、自己が個体として連続しつづけ、同じ人格であるのだ。
\section{
意識にとって連続とはいかなることか
}
しかし、意識が問題であるとき、この\kenten{連続}ということの内実はそう明らかではない。これまで問題にしてきた種類の事物では、原子にしろ物体にしろ動植物にしろ、連続とは時間的に連続していることで、ある時点から別のある時点まで途切れずに存在しつづけることだと考えてきた。しかし、意識はこのような意味では連続していない。なぜなら、ロックも認めているように、すくなくとも眠っているあいだ、人は意識をしていないからだ(2.1.11)。人は起きているときはつねに意識している。そして眠る時には意識をやめる。そうだとすれば、人の意識は毎日新たに始まって、毎日終わっていることになる。この場合、連続性の原理、あるいは同じことだが単一起源の原理からして、これらの毎日の意識は始まりが異なる別個の意識であり、それゆえ、一日として同じ人格は続かないことになる。これはあまりに受け入れがたい結論だ。
\par
しかしながら、ロック人格同一性の検討において、連続性の原理を捨ててはいない。ロックは次のように語る。
\begin{quote}
\kenten{自己}についてのこうした説明では、数的に同じ実体が同じ\kenten{自己}を作るとは考えられていない。そうではなく、同じ\kenten{連続した}〔強調は小林〕意識が〔同じ\kenten{自己}を作ると考えられているのだ〕。(2.27.25)\footnote{ここでは連続した意識が作るのは「自己」であると言われているが、この自己は人格と同一なものと考えて論を進める。これは従来主流の解釈であるが、本論考第3章で否定される。}
\end{quote}
このロックの言い方を認めるとすれば、ロックはどこに意識の連続性を見ていたのだろうか。
\par
空間的連続性は役に立たない。ひとりの人間は眠っている間でも、その位置は空間的には連続しているが、これはあくまで\kenten{人間}の同一性の根拠になるだけにすぎない。というのも、意識は眠っているあいだそもそも作用していないので、ないものに空間的位置があるとしてその連続性を考えるのは不合理だからだ。それでは、他にいかなる連続性を考えればいいのだろうか。
\par
連続という語で、今まで考えてきたこととは別のことが意味されているのかもしれない。ロックは『知性論』のある箇所で、「推論の連続した連鎖」(4.12.7)という用法で‘continued’の語を用いる。推論の「連続性」が、時間の連続性と同じ意味であるとは考えづらい。「推論の」として用いられる場合、連続という語の意味は「いかなる間断もなく」というよりはむしろ単に「つながりがある」程度の意味で用いられている。意識は時間的に間断ない仕方では連続しつづけることがない。しかし、意識の連続性を、\kenten{それぞれの意識の連続している部分のあいだにあるつながりのようなものだと考えれば}、説明がつくのではないだろうか。その場合、何がある時点での意識とほかの時点での意識のあいだをつなげるものなのだろうか。それは、同じ身体を持つことではない。そのことを示したのが、有名な王子と靴直しの思考実験である。
\begin{quote}
ある王子のたましいが、万一、その王子の過去の人生の意識をそれとともに運んで、ある靴直しの身体に、そのたましいが離れてすぐに入り、活気付けたとしたら、誰もがそう考えるが、彼は王子と同じ人格であり、その王子の行動にのみ責任を負うのだ。(2.27.15)
\end{quote}
この思考実験では王子の意識が靴直しの身体に入ってもなおその人格は王子と同じものであると想定されている。直前の引用で確認した通り、人格が同じであるのは、意識が連続しているからである。ここから、意識を連続させるのが身体の連続性でないことは明白だろう。
\par
そして同じ思考する実体(たましい)において思考することが意識を連続させると考える必要もない。これに関しては、昼人間と夜人間の事例から確認できる。
\begin{quote}
もしわれわれが同じ身体〔で〕作用する二つの異なる、伝達不可能な意識があって、一方はつねに日中に、他方はつねに夜間に〔作用する〕と想定できるならば、〔…〕私は問うが、最初の場合、\kenten{昼人間}と\kenten{夜人間}は、\kenten{ソクラテス}と\kenten{プラトン}と同じように異なる二つの人格ではないだろうか。〔…〕上記のこの\kenten{意識}が同じまたは異なるのは、それ〔意識〕をそれらの身体に持っていく同じまたは異なる非物質的実体〔すなわち、たましい〕のおかげであると言うこともまた、重要ではない。〔…〕というのは、〔…〕明らかに、非物質的な思考する事物は時折その過去の意識を手放して、〔のちに〕再びそれを復帰することがありえるからである〔…〕記憶と忘却のこの間隔を、それぞれの番が日中と夜間と規則的に回ってくるようにすれば、同じ非物質的精神を持った2人の人格が得られる〔…〕。それゆえ、\kenten{自己}はそれ〔自己〕が確信しえない〔物体的あるいは精神的〕実体の同一性や差異性によってではなく、意識の同一性のみによって決定されるのである。(2.27.23)
\end{quote}
ロックは、意識の連続性の内実が身体の同一性であるという意見にも、たましい(思考する実体)であるという意見にも賛同しない。あくまで彼は、意識それ自体において、意識の連続性が成り立つと考えているのである。
\par
それでは、何がこの意識のつながりをつくるものとして考えられるのか。何によって、過去の意識の連続と現在の意識の連続とは、そのあいだにギャップがあるにもかかわらず「同じ意識」となるのだろうか。時間的連続性も空間的連続性もといった客観的ですでに実際にある連続性が役に立たないのだとしたら、なんらかの主観的な精神的作用が意識をつなげ、連続性を構成すると考えるほうがよさそうである。もっとも考えやすいのは、\kenten{記憶}がそれ自体は連続していない意識をつなげるという考え方、すなわち、記憶説である。
\section{
記憶説
}
この記憶説は昔から、トマス・リードやジョセフ・バトラーらによって採用されてきた伝統的な解釈であり、また現在の解釈にとっても、スタンダードな考え方である\footnote{たとえば、Flew(1956)やDicker(2019)は記憶説でロックの人格同一性論を解釈する。}。記憶説はたいてい、次のように定式化される。
\begin{itemize}
\item[]$ある時点t_1における人格P_1が、それより後の時点t_2における人格P_2と同一である$
\item[$\iff$]$時点t_1における人格P_1が持つ意識C_1が、時点t_2における人格P_2が持つ意識C_2と同一である$
\item[$\iff$]$時点t_2における人格P_2が、時点t_1における人格P_1のなした経験・行為・考えの$\kenten{記憶を持っている}
\end{itemize}
\par
ロックのテクストにはこの記憶説をとる根拠がいくつも存在する。ロックは同一性章において、意識の通時的同一性が記憶に存すると述べているようにみえる表現をいくつもしている。「この意識が後ろ向きに何かしらの過去の行動や思考に広がりうる限り、それだけかの人格の同一性も達する」(2.27.9)や、「その同じ意識は過去の行動の現在の表象」(2.27.13)や、「この意識は、常に忘れやすさによって妨害されている」(2.27.10)などがある。あるいは、次の引用も強力である。
\begin{quote}
ある知的存在者がある過去の行動の\kenten{観念}を、その行動について最初に持ったのと同じ意識、そして現在の行動について持つのと同じ意識とともにくり返すことができる限り、その知的存在者は同じ\kenten{人格的自己}({\itshape personal self})である。(2.27.10)
\end{quote}
ここで過去の観念を繰り返すというのは、ロックにおいてはそのまま「記憶している」と表現されることがらである。
\begin{quote}
保持のもうひとつの仕方は、刻印のあとで消えてしまった、あるいは、いわば視界の外に置かれた\kenten{観念}をわれわれの心の中で再び復活させる力である。〔…〕これが\kenten{記憶}である。(2.10.2)
\end{quote}
観念を「再び復活させる力」がある(すなわち記憶している)ことと「くり返すことができる」ことは同じことだと考えられる。このように、記憶説を支持するように見える箇所が同一性章には多い。
\par
この記憶説をとった場合、意識は時間的に連続してないことを認めても、意識の同一性を主張することができる。なぜなら、この場合、ある時点$t_1とそれより後の時点t_2$のあいだ意識が時間的に連続している必要はなく、ただ、$t_1における思考や行動をt_2$においても覚えていれば、それが意識の』\kenten{連続}(ロックの連続にはつながり程度の意味もあることは先ほど確認した)を構成し、それで時点$t_2における意識C_2は時点t_1における意識C_1$と同一の意識と言ってよいからである。こうして、記憶説は先ほどの連続性説よりも耐久性のある議論だといえよう。
\par
しかし、記憶説にはさまざまな困難も挙げられている。第一にテクスト的批判を紹介しよう。問題は二つある。ひとつめは、意識の同一性が記憶に存するわけではないことを示唆する箇所が、同一性章に存在することだ。同一性章10節では「その同じ意識が過去の行動や来るべき行動に及ぶことができる限り、その知的存在者は同じ\kenten{自己}であるだろう」(2.27.10)と述べられるが、ここでは、\kenten{未来}の人格との同一性が成り立つことが示唆されている。未来の人格と現在の人格をつなげるものは、記憶ではありえない。ふたつめは、アサートンの指摘するように、ロックがその人格同一性論の要点を述べる9節と27節で、「記憶」という言葉が全く使われないことである\footnote{Atherton(1983), pp. 276-277.}。ロックにおいて記憶が直接に人格同一性を構成していたとするには、いまひとつ証拠が足りないのである。
\par
ほかにも記憶説には、バークリーによって考案され、リードによって定式化された「勇敢な将校の事例」と呼ばれる反論がある。現代の論者の多くが、この反論から導かれる記憶説の重大な難点は、意識説で考えられた人格同一性と、推移律で考えられた人格同一性とが相克するという点だと見ている\footnote{たとえば、Kaufman(2016), pp. 253-254.}。簡単にリードの議論を再構成してみる。
\par
次のような人がいたとする。その人は子供のころ、果物泥棒をして、その罰としてぶたれた。そしてその人は青年のころ、将校になり、初陣で敵軍の軍旗を取った。さらに長じてその人は壮年のころ、将軍に任命された。続いて次のことを仮定する。その将校は、子供のころ果物泥棒をしたことを覚えていた。そして将軍は、青年のころ軍旗を取ったことを覚えていた。しかし、将軍は子供の頃果物泥棒をしたことを覚えていなかった。これらのことから、次のことが帰結する。まず、記憶説に従えば、将軍は果物泥棒をしたことを覚えていないので、将軍は果物泥棒をした子供と同一人格ではない。しかし、記憶にもとづく人格同一性が推移律を満たすとすれば、将軍は果物泥棒をした子供と同一人格となる。なぜなら、将軍は将校のころ軍旗を取ったことを覚えているので将校と同一人格であり、かつ軍旗を取った将校は子供のころ果物泥棒をしたことを覚えているので将校は子供と同一人格であって、ここから推移律にしたがって、将軍は果物泥棒をした子供と同一人格だと言えるからである。こういうわけで、人格同一性、あるいは意識の同一性の記憶説を取った場合、その同一性という関係に推移律は成り立たなくなるのだ。
\par
さらに記憶説には、ジョセフ・バトラーが提示した有名な反論がある。次の引用が有名である。
\begin{quote}
人格同一性の意識は人格同一性を前提していて、それゆえ、知識が、それ〔知識〕が前提している真理を構成することはどんな場合でもできないのと同じように人格同一性を構成できないということをひとは自明だと実際考えるはずである。\footnote{Butler(1736), from ed. Perry(1975), p. 100.}
\end{quote}
\par
$
バトラーの反論は、ロックの議論の循環を指摘するものだとマッキーはいう\footnote{Mackie(1976), pp. 186-187.}。まず、ふつう、ひとは自分以外の人格の行為の記憶を持ちえない。すると、ある人格P_1が記憶している行為は当然P_1$\kenten{自身が為した}行為であるはずである。そうだとすると、記憶にはすでに\kenten{自分の人格の}記憶であることが先取されていることになる。ゆえに、記憶によって人格同一性を構成することは不可能なのである。この批判は強力な批判である。
\section{
専有説
}
記憶説に対するこれらの問題を解決するような解釈が、ビーハンやウィンクラーによって進められたこの専有(appropriation)説である。これによれば、意識の同一性、ひいては人格の同一性の定式化は以下のようになる
\begin{itemize}
\item[]$ある時点t_1における人格P_1が、それより後の時点t_2における人格P_2と同一である$
\item[$\iff$]$時点t_1における人格P_1の持つ意識C_1が、時点t_2における人格P_2の持つ意識C_2と同一である$
\item[$\iff$]$時点t_2における人格P_2の意識C_2が、時点t_1$における経験・行為・考え(知覚)の記憶を持ち、その経験・行為・考えを、現在専有している
\end{itemize}
経験・行為・考えを専有しているとは、どういったことか。一言で言えば、その経験・行為・考えが、みずからの快苦に影響するかどうか、影響するならばどのようにかを気にかけている(be concerned)ということである。なぜ、意識の同一性を専有だと考えることができるか、理路を辿っていこう。
\par
まずは専有説における「専有」の内実を見てみよう。第一歩として、ロックにおける意識と所有あるいは帰属の関係を辿っていく。まず本章1節で述べたように、意識は「世界のある部分を自分自身の部分だと認識する、あるいは『考える』精神的作用」\footnote{Yaffe(2007), p. 214.}であったことを思い出されたい。そして意識による帰属は、帰属するものへの気がかりを伴う。
\begin{quote}
人の身体の四肢は、だれにとっても\kenten{その人自身}({\itshape himself})の一部で、その人はそれらに共感し、それらを\kenten{気にかける}(is concerned for them)〔強調は小林〕。手を切り落とされ、それによって手をわれわれがその熱さ、冷たさ、そして他の感発物に対して持っていたその\kenten{意識}〔強調は小林〕から分離すると、そのとき、もはや\kenten{その人自身}のではなくなり、物質の遠く離れた部分と変わらない。(2.27.11)
\end{quote}
これらをまとめると、事物(ここでは手)は、意識とともに自己の一部になり、その事物はそのひとによって気にかけられるということになる。さて、このように意識が人の自己に身体を帰属させるわけだが、同じ仕方で人は身体以外の事物をも自己に帰属する。
\begin{quote}
私は千年前になされて、この自己意識のために今私に専有されているある\kenten{行為}〔強調は小林〕を、一瞬前にしたことと同じくらい気にかけ、それに正当に責任を持つ。(2.27.16)
\end{quote}
このようにロックは意識の気にかけ(concern)によって帰属するものとして、過去の行為も挙げているのである。そして、過去の行為のこうした帰属によって、ひとは過去の行為の責任を正当に持つのである(2.27.18)。このような種類の所属、すなわち、過去の行為の意識による自己への帰属で、それによってその行為をひとは気にかけ、正当に責任を持つ、そういった帰属が専有である。このことからビーハンは、この専有を、帰責を可能にする条件であると述べている\footnote{Behan(1983), p. 69.}。
\par
ここまで説明してきた専有説には二つの長所がある。ひとつめは、ロックの法廷用語説をよく解釈できること、そしてふたつめは、先ほど記憶説の解説で紹介したリードとバトラーの批判に応答できることである。まず、この長所について見てみよう。専有説では、ロックの法廷用語発言をよりよく解釈することができる。法廷用語発言とは、同一性章26節における、「それ〔人格〕は〔…〕法廷用語(Forensick Term)である」(2.27.26)という発言である。この発言をどのくらい重要視するかに関しては意見が分かれる\footnote{例えばDicker(2019)はそこまでの重要性を置かないが、一方で、一ノ瀬(1997)はこれを人格概念の中心に据える。}。記憶説にもとづく人格概念理解では、人格が法廷用語であるべき必然的な理由は見つからない。せいぜい、記憶によって同一になる人格という主体を帰責主体とすることはできなくはないという程度のことしかいえないのである。これに対して、気がかりという契機を含む専有という作用は、ビーハンが示す通り、それ自体で帰責を可能にする条件である。よって、専有説ならば、人格概念が法廷用語であることの理由は概念それ自体に内在的に説明できるのだ。このように、専有説にはロックの法廷用語発言のよりよい解釈を与えるというメリットがある。
\par
続いて、ふたつめの長所について見てみよう。この専有説解釈は、先ほど示したリードの批判にどのように応答することができるのだろうか。ヤッフェは、この専有説であれば、推移律は守られるのだと主張する\footnote{Yaffe(2007), pp. 221.}。その理路はこうだ。
\par
$
ヤッフェは専有という関係を、推移律を充足するものとして定義する。彼は専有という関係には、直接専有と間接専有があると論じ、これを説明する。直接専有とは、ある時点t_2における人格P_2が、ある時点t_1における人格P_1によってなされた行為A_1を「私がした、あるいはしている」と意識的に自覚することである。そして人格P_2が人格P_1によってなされたある行為A_1を直接専有していて、かつ、その人格P_1がその前のある時点t_0において人格P_0がなした行為A_0を専有しているとき、人格P_2は行為A_0を直接的に専有していない場合でも、間接的に専有しているということができるのである。これと同じような仕方で$\kenten{間接記憶}を定義したとしても、それはもはや\kenten{記憶}の名に値しないが、専有の場合はこのような\kenten{間接専有}も、きちんと\kenten{専有}の名に値するのだとヤッフェは論じる。
\par
先ほどの勇敢な将校の例にあてはめると問題は次のようになる。子供の頃の盗みを忘れた将軍が、彼が将校だったときにその盗みのことを覚えていたという事実を伝えられたとする。このとき将軍は、その少年時代の盗みのことを、たとえ記憶はなかったとしても、自分の行為だと自覚することができるはずである。なぜなら、将校は、自分がかつて兵士であったことの自覚があり、その兵士がかつて盗みを働いた少年のなしたことを専有していたのが事実なら、自分はその少年であったのだと「論理的に」考えざるをえないからである。ヤッフェはこのことを、「同一性という考えによって」専有するべきという「理性的圧力」が働くというふうに表現する\footnote{Yaffe(2007), pp. 222.}。そしてこれによって、将軍は子供のころの盗みを自分がした行為だとして\kenten{間接的に}専有することができる。この自覚は、記憶とはことなっていて、間接的であったとしてもなお専有よばれる資格がある。このことから、ヤッフェはこの仕方で定義された人格同一性は推移律を満たすと述べる。
\par
$
次に、バトラーの批判、すなわち記憶説に対する循環の反論については専有説はどのように応答できるのだろうか。バトラーの批判は、記憶は過去の当人の記憶以外にありえないので、記憶に残っている行為をしたの人格が当人であるとして人格同一性を構成するのは論点先取だというところにあった。これに反論するには、ある人格が、過去に同じ人格が実際にはしなかった行為でも専有しうることが示さればよい。それが言えたならば、専有していることそれ自体に人格同一性が前提されているという批判はあたらなくなるからだ。そして実際、私は過去に私がしなかった行為でも専有しうるは。たとえば、わたしが帰省して久しぶりに親戚に会ったとき、自分が5才のときに、遊園地のパレードに乱入して大騒ぎを起こしたことがあったと複数の人に聞かされたとしたら、そしてその時のことを私は自分が記憶している12才のころ覚えていたと聞かされたならば、私自身にまったくその記憶はなくても、その行為を専有するだろう。そして、その専有はその行為を事実私がしていたかどうかには関係がない。このように、ひとは過去の自分が実際には行っていない行為を自分がしたものとして専有することがありえるため、専有に人格同一性は前提されておらず、よって、循環の批判を免れることになる。
$
\par
しかし、バトラーの問題をこの形で回避することは、専有には正しい専有と間違った専有を区分する原理がないということをも含意することになる。これによって、他の問題が姿を表す。それは、意識内容が事実に反する形で人格同一性を規定する可能性があるという点である。つまり、自分がある行為をしたと考えるだけで、そのことによって本当にその行為をした人格になってしまうということになりかねないのである。このように、思考だけで人格同一性が事実に反した形で成立してしまうことの危惧を免れえないことは錯誤の問題と呼ばれ、多くの論者が指摘している専有説の難点である\footnote{たとえばYaffe(2007), p. 223. や一ノ瀬(1997), p. 116. など。「錯誤の問題」という表現は一ノ瀬に負っている。}。これでは、専有していることは人格の同一性の規準(criterion)にはなりえないだろう。なぜなら、実際に罪を犯した者が罰されるのではなく、罪を犯した\kenten{と思っている}者が罰されることになるからだ。これは到底受け入れがたい示唆である。ロックの人格同一性論はどのようにすればこれを回避できるのであろうか。
\par
果たして、問題はどこに存するのか。この問題は、ひとえに、専有を一人称的で私秘的な心理作用だととることに起因する。というのも、一人称的で私秘的な心理的言明は、偽であることができないからである\footnote{私が念頭に置いているのは、シューメイカーの「一人称の心理的言明」(first-person psychological statement)に関する議論である。シューメイカーは、「私は歯が痛い」「私は雨が降るだろうと思う」などの、一人称がその心に何かしらを思うことを言明する命題は、実際その人がそう思っているだけで真であり、その真偽を判定するのに、当人にとって客観的な事実(たとえば、病理学的に虫歯といえるものが実際にがあるかどうか)を確認する必要がないし、仮に事実が一人称の心理的言明に反していたとしても、その言明を棄却する必要はないのだと述べる。 cf. Shoemaker(1963), pp. 16-17.}。専有が一人称的で私秘的な心的作用あるいは言明であるかぎり、事実に照らすというステップはその言明が正しくなされるために必要ではないのだ。事実に照らそうが照らさまいが、自分がその信念を持っている限りそれは間違いになりえないのである。問題になっているのは、客観的な事実性である。一人称的に信じられる専有によっては、観念的な同一性は構成されるけれども、事実的な同一性は構成されないのである。これは記憶説まで遡って、人格同一性が一人称的に構成されると考える際に避けられない問題である。
\section{
三人称的構成説
}
いま述べた問題、すなわち、考えるだけで実際に人格同一性が成り立ってしまうという不合理を回避するための策は、人格同一性が第三者によって、三人称的な視点から構成されるものとして考えることである。この解釈を、三人称的構成説と呼ぶこととしよう。一ノ瀬\footnote{一ノ瀬(1997)}がこの三人称的構成説の論者である。
\par
人格同一性、あるいは意識の同一性を第三者が三人称的に構成するという見方は、リードの批判、バトラーの批判、そして、先ほどの錯誤の問題を一挙に解決できる見方である。リードの批判に対しての応答は次のようになる。すなわち、将軍が子供時代の果物泥棒を忘れていても、専有できなかったとしても、証拠をもとに事実を吟味して第三者が同一性を決定するかぎり、その同一性は推移律を満たしているものになるはずだ。バトラーの批判に対しての応答は次にようになる。すなわち、ある人格の同一性が第三者によって三人称的に構成されるのであれば、そこに当人の人格の同一性が先取されているはずはない\footnote{一ノ瀬(1997), p. 160.}。そして、錯誤の問題に対しての応答は次のようになる。すなわち、三人称的な同一性の構成においては、例えば法廷などで証拠を出し合って吟味した上で議論し判断するという手続きを踏むことで、事実を参照する形で同一性が判断されるため、この手続きが真なる同一性の主張と偽なる同一性の主張とを見分けることができる。
\par
このように、三人称的構成説によって、ここまで見てきたロックの人格同一性論に関する多くの問題に応答することができる。とはいえ、ロックの人格同一性論の解釈者は伝統的に、意識の同一性が一人称的に構成されるものだとして考えてきた。それでは一ノ瀬は、どのような根拠からこのようになラディカルな読み替えを敢行するのであろうか。
\par
一ノ瀬が根拠として重要視するのは、先ほども触れたロックの「それ〔人格〕は〔…〕法廷用語である」(2.27.26)という発言である。伝統的な解釈に従うと、この発言でロックが主張しているのは、一人称的に構成された人格が、法廷や道徳に関する実践的場面で行為主体として考えられるところのものである人格でもあることにすぎない。しかし、一ノ瀬はさらにこの表現を読み込み、法廷用語であることが人格概念にとって第一の特質であると考える。そのことから、意識および人格の同一性は三人称的な実践の場面、例えば法廷のような場面第三者によって決定されるのが本来のあり方だと結論づける。
\par
しかしながら、この一ノ瀬の解釈にはロックのテクストと整合しない点がいくつかある。以下に二つの難点を挙げよう。まず一点めとして、ロックのいう意識が三人称的に決定されてよい理由の説明が不十分である。ロックが「私がいまなんらかのものを意識していると私自身が知覚していないとき、ほかの人は私が今何らかのものを意識していると知覚することができるだろうか〔、いやできない〕」(2.1.19)と語っていることからもわかるように、意識は『知性論』の議論においてはあくまで私秘的な知覚として考えられてきた。それにも関わらず、一ノ瀬は、意識の同一性の決定には、第三者がその最初の権威をもつと述べる\footnote{一ノ瀬(1997), p. 113-114.}。これを一ノ瀬の理路に即して説明すれば次のようになる。まず、一ノ瀬は人格の決定と成立を区別している。彼によれば、人格は第三者に\kenten{決定}されたのち、それに当人が同意するという\kenten{決定}という二段階の決定をへて人格が\kenten{成立}する\footnote{一ノ瀬(1997), p. 174.}。ここでは、$今の人格は過去に行為A_1, A_2, \ldots をした人格であるというふうに$、複数の行為を集めて同一人格を最初に\kenten{構成}するのは、法廷における第三者である\footnote{「差し当たっては、法や道徳を探究し同意決定していくのは第三者であって、そのことによって判定の主題とされている者に人格概念が適用されていくという、そうした構造になっている」一ノ瀬(1997), p. 168.}。そして、当人は第三者が構成した人格を追認する形での二次的な権威をもつにとどまるのだ。この議論において一ノ瀬は、意識の最初の権威が第三者になければならない必要性があることを論じているが、意識が三人称的権威によって最初に決定されるという解釈をロックが\kenten{許すかどうか}については管見の限り考慮のあとが見当たらない。そして私は、どうして意識が三人称的に決定される解釈されて\kenten{よい}かを明らかにするまで、この解釈はロックから離れた解釈であるとの批判を免れえないと考える。
\par
次に二点めとして、人格が道徳的な概念であることの理由を、一ノ瀬の三人称的構成説は読み違えているように思われる。一ノ瀬の考えでは、人格が実践的な場面において、決定・同意といった行為を通じて三人称的に構成されることで道徳的主体として使用される権利を得る。\footnote{cf. 一ノ瀬(1997), 第6章3節}。しかしこれでは、ロックにとって重大な関心事だったはずの最後の審判の場面で人格概念が帰責主体であるべきことの説明ができない。\footnote{ロックは同一性章22節、26節で最後の審判を話題にあげる。最後の審判である主体が罰を受ける際、その主体がその罰に値する行為をしたことを意識していなければいけないというのが彼の主張である。}を考えるならば、神によるこうした決定・同意の行為が人格概念を道徳的な概念にすることになるはずだ。しかし、「無限に賢明な」(infinitely wise)神には、決定・同意の行為実践そのものが不要なのではないだろうか。神の前でも人格が帰責主体である限り(2.27.26)、その道徳性は、人格が法廷で処罰に関する決定行為に巻き込まれていること以外の源泉から付与されているのではないだろうか。そして同一性章を見る限りでは、その源泉は先ほど述べられた専有だと考えるのがもっとも説得的であるように思われる。
\par
さて、ここまで三人称的構成説の議論に対する批判を述べてきたが、しかし、この解釈がロックの人格同一性論をうまく説明するひとつの道だということに変わりはない。人格同一性をこの三人称的構成説で捉えることを拒絶するとすぐに、人格同一性のうちに、客観的にみて数々の矛盾が生じてしまうことは先に見た通りである。そしてのちに詳しく検討するが、ロック自身が三人称的な場面で正当に決定される人格同一性があることを明白に述べている「酔漢の事例」と呼ばれる箇所が同一性章22節にはある。
\par
一人称的構成説と三人称的構成説の対立が反映しているのは、ロックの人格同一性論に内在する二つの視点の対立である。今村はこれをロックの人格同一性論における理論的部分と実践的部分の対立と特徴付け\footnote{今村(2010), pp. 30-31}、一ノ瀬は意識説と法廷用語説の対立と捉える\footnote{一ノ瀬(1997), p. 155.}。ロックの人格同一性論を統一的に理解するには、この対立を調停することが求められる。ロックの人格同一性論におけるこの二つの、一見互いに排斥するような側面を、単なる矛盾であると断ずるよりほかにどのように受け取ることが可能なのか。解決の糸口はどこにあるのだろうか。
\chapter{
人格は自己に対する名である
}
前章の最後で提示された問題を再定式すると以下のようになる。すなわち、ロックの人格同一性論において、同じ意識を構成するのが一人称的で私秘的な心理的作用だとすると、人格同一性は事実に合致していなくとも、覚えていたり、気にかけたりする一人称的な心理作用だけで成立してしまう。他方で、人格同一性が実践的な場面で第三者によって構成されるもので、意識の同一性も三人称的な権威にもとづくと解釈すればこれらの問題はなくなる。しかしなぜ意識が三人称的に決定されてもよいのか、その点はまだ明らかになっていない。またそのようにして人格同一性が構成されたとしても、そうした人格概念が最後の審判で帰責主体としてはたらくことを内在的に説明できない。
\par
問題の根底は、人格同一性が事実に合致することの保証と、人格概念が法廷用語であるための権利の保証がトレードオフの関係にあることだ。よって、最も良い解決策があるとすれば、それは、三人称構成説によってもたらされる事実への対応と、一人称的構成説によってもたらされる法廷用語としての権利の説明を二つ一度に満たす説明方式を見つけることに違いない。そしてロックはそういった解釈に至る手がかりを同一性章に記している。
\par
その手がかりが、「\kenten{人格}は〔…〕この\kenten{自己}に対する名である」(2.27.26)という一節である\footnote{今後この命題を完全な引用の形でなく「人格は自己に対する名である」などと用いることがある。}。なぜこれが人格の一人称的側面と三人称的側面をつなぐとできると期待できるのか。それは「自己」が一人称に深く関わるものであり「名」は三人称的他者との言語実践に深く関わるものだからである。
\par
自己が一人称に深く関わることは、自己が一人称的な意識の作用によって構成されることから明らかである。このとき、自己と人格とはどのように関係しているのだろうか。直観的には、自己と人格のあいだには何の違いもないと言ってみたくもなる。実際、管見の限り両者の違いに注目した論文は見られなかった。さらにロックのテクストにも、人格と自己は双方互換可能であるかのように映る書き方がなされているところがある\footnote{たとえば、2.27.16の欄外要約には「意識が同じ人格をつくる」とあり、2.27.23の欄外要約には「意識のみが自己をつくる」とある。}。しかし、のちに論じることになるが、詳細に検討してみると、ロックは人格の概念と自己の概念を区別して使用している。その場合、人格が持っているように思われていた一人称的な側面は、本当は人格ではなく自己が持っている側面であった可能性を考えなければならない。こうした背景から、本章で私は最初に人格と自己の関係を問う。
\par
他方で、「名」が三人称に深く関わることは、自己が一人称に関わることほど容易には了解できないかもしれない。しかし、ロックによればことばとは、互いの心のうちの私秘的な観念を(知性の限界に)直接は伝達できない人間たちがそれでもコミュニケーションをその目的としていることに存する(3.2.1)。ここでいう名とは、ほとんど名詞という意味である。人格が名詞として言語実践の中で何かを意味し、伝達することの中に、人格を三人称的に構成されたものとしてとらえなければならない理由が析出してくる。
\par
こうした背景から、本章では人格と自己の関係を問うたあとに、名についてロックが何を述べているかを簡単にまとめたうえで、「\kenten{人格}はこの\kenten{自己}に対する名である」という発言を解釈する。そしてそれによって得た解釈枠組みによって、先ほど問題となった意識の同一性の一人称的構成説と三人称的構成説を仲介した、新たな解釈を提示する。
\section{
自己と人格の違い
}
最初に、自己と人格の概念を、ロックは区別して用いていたということを示したい。先ほども述べたように、これまで多くの論者は人格と自己という概念が同じだと考えてきた。これについては、人格という概念が同一性章ではじめて定義される9節を改めて検討し、そこで出てくる「自己」が「人格」とは別であったことを示すこととする。
\begin{quote}
(1)私たちは\kenten{人格}が何を意味するのか考えなければならない。〔人格が表すのは、〕思うに、理性と内省をもち、それ自身をそれ自身だと、〔つまり〕異なる時間と場所で同じ思考する事物(thinking thing)だと考えることができる思考する知性的存在者(Being)である。(2)〔人格〕はそれ〔それ自身をそれ自身だと考えること〕を、思考から切り離すことのできないあの意識のみによって行う。そして意識は、私の見るところ、思考にとって本質的(essential)である。だれもが、自分こそがまさに知覚しているということを知覚せずには知覚することができない(It being impossible for any one to perceive, without perceiving, that he does perceive)からである。何かを見るとき、聞くとき、味わうとき、感じるとき、熟考するとき、あるいは意志するとき、われわれは自分がそうしていることを知っている。このように、それ〔意識〕とは、つねにわれわれの現在の感覚や知覚に関するものなのであり、これによってすべてのひとは彼〔そのひと〕自身にとって、彼〔そのひと〕が\kenten{自己}({\itshape self})と呼ぶものなのである。(3)ここでは、同じ\kenten{自己}が同じ実体の中で継続するか、あるいは異なる実体において継続するかのどちらであるかは考えられていない。というのも、意識(consciousness)は常に思考に同伴しており、そしてそれ〔意識〕が、全ての人を彼が\kenten{自己}({\itshape self})と呼ぶものにするものである。そして、このことによって(thereby)、〔意識は〕彼自身を他のあらゆる思考する事物から区別する(distinguish from all other thinking things)。ここにのみ、\kenten{人格同一性}、すなわち理性的存在者の同じさ(sameness)は存する。〔…〕(2.27.9(1)から(3)の番号は小林)
\end{quote}
この節を引用中に示したように分けて、それぞれの内容を検討していこう。
\par
まず、(1)は人格の定義である。ここで人格の特徴として述べられているものは理性的存在者という伝統的な人間存在理解と重なるが、そのなかでひときわ目を引くのが「それ自身をそれ自身と、すなわち同じ思考する事物だと、異なった時点と場所で考えることができる」という要件である。これは通時的な自己同一性の認識の可能性が人格の要件に含まれることを示している。
\par
この認識がいかになされるのかが、続く(2)で述べられる。それによれば、通時的自己同一性の認識は意識によってなされる。第2章で触れたように、意識の内実が、「自分が知覚していることを知覚」することである限り、一人称的に「自分」と呼ばれるようなある主体を想定する。この「自分」にあたるものがここで\kenten{自己}と呼ばれているものである。意識はこの自己に知覚、思考、行為その他を結びつける。こうした自己の構成のことをウィンクラーは「主観的構成」と述べる\footnote{Winkler(1991), p. 201.}。
\par
そして、(3)で重要なのは、人格の同一性よりも先に、自己の同一性が語られていることである。「意識(consciousness)は常に思考に同伴しており、そしてそれ〔意識〕が、全ての人を彼が\kenten{自己}({\itshape self})と呼ぶものにするものである。そして、このことによって(thereby)、〔意識は〕彼自身を他のあらゆる思考する事物から区別する(distinguish from all other thinking things)。ここにのみ、\kenten{人格同一性}、すなわち理性的存在者の同じさ(sameness)は存する」(2.27.9)を見れば、そのことは明らかだろう。意識によって作られるのは第一に自己なのである。
\par
人格概念と自己概念の出自を整理すると次のようになるだろう。まず最初にロックによって言及されているのは人格概念である。人格概念とその同一性は同一性章以前からひとつの問題として扱われていた(1.4.5)。これに対して、自己概念はこの人格の概念を説明する際に初めて明示的に導入された。それは第一には「それ自身をそれ自身と異なった時点と場所で考えることができること」という人格のひとつの契機であり、第二に、その認識に必要な再帰的知覚としての意識が構成するものである。この意味での自己、すなわち、意識によってすべての心的作用の主体とされ、同時に個体化された自己を、ロックは同一性章を通じて‘{\itshape self}’とイタリック体で表記し、単に再帰代名詞としての‘代名詞+self’と区別しているように思われる。この主体としての自己は、この議論において人格概念を基礎づけていて、さらに人格同一性の原理としても挙げられている。このことから、ロックがこれを少なくとも人格と同じ概念として使用していないことは明らかだろう。
\section{
自己に対する名としての人格
}
それでは、「\kenten{人格}は〔…〕この\kenten{自己}に対する名である」({\itshape Person} ... is the name for this {\itshape self})という命題の解釈に入りたいところだが、その前に、軽く「名」(name)とはなにかを確認する。まず、名はことば(words)のひとつである。そしてことばとは、発話者の心のうちにある観念を示す記号であるような特定の分節音である(3.2.1)。この名には固有名と一般名の二つがある。固有名とは、ただひとつの特定の事物を意味表示する名であり、一般名とは、多数の特定の存在をそれひとつで表示する名である(3.2.3)。一般名が多数の特定の存在を意味表示することができる(一般的である)のは、それがひとつ以上の個物を表象できる観念、すなわち抽象観念の記号であることによる。抽象観念は、ひとが個々の特殊な観念から、それをある特定のものに規定する観念(時間や空間や固有の性質など)を抽象することによって作り出すものである(3.3.6)。特定の事物(の観念)は、この抽象観念と一致すると見られることによって「種(sortあるいはspecies)」の一般名で呼ばれる。すると、特定のものを一般名で呼ぶことは、その特定物をその種に分類することになるのだ(3.6.43)。たとえば、「この馬」というだけで、そこで指示された特定物が「馬」という種に分類されることになる。このとき、「馬」という種に属する事物(の観念)のそれぞれは、すべて馬の抽象観念に一致するものであるために、馬という種に分類されるのだ。
\par
それでは、実際に「\kenten{人格}はこの\kenten{自己}に対する名である」という命題の解釈をしていこう。最初に、この命題で使われている、「人格」「自己」という語が何を表しているかを考えなければならない。というのは、先ほどの名についての考察を踏まえれば、これらが表すものの候補として、名、種の抽象観念、名に結びつけられる種の個々のものの少なくとも三つがありえるからだ(例えば「馬」と言われた時には、「ウマ」という分節音、あるいは、四足歩行で、ひづめを持ち、その他これこれの特徴をもつ動物という「馬」の抽象観念、あるいは、馬である任意の個物のそれぞれがその指示対象として考えられるだろう)。以下では最初にこの難点を解決したい。
\par
さて、当該命題中で「人格」は何を表しているのだろうか。これは述語‘is the name’の主語であることから、これは「人格」という名そのもののことを表していると考えてよいだろう。次に「この\kenten{自己}」(this {\itshape self} )が何を表しているかを考える。ここで‘this’という語が入っているのは、この‘self’が前の文脈で語られた‘self’と同じ仕方で使われていることを示している。そこで直前の節を見てみよう。25節の最後は、「以前の行為の意識によってそれ〔現在の思考する事物〕に合一する事物はなんでも、その時と今とで同じである\kenten{同じ自己}の一部をなすのである」(2.27.25)と結ばれている。この文脈を踏まえるならば、件の命題で「この\kenten{自己}」が表すのは任意の個人であり、自己という種に分類されているものということになるだろう。そう考える理由は次の通りである。まず「この\kenten{自己}」は、名、名が表す種の任意の個物、名が表す種の抽象観念のいずれかである。この候補のうち、最初に除外されるのは、「自己」が名それ自体を表しているとする選択肢だろう。というのも、そもそも名は、そのものとしては分節音でしかないことを踏まえれば(3.2.1)、件の命題の解釈は、ジコという分節音を先の「人格」が名すなわち記号として表しているということになり、これはあまりに奇異な解釈だからである。次に、自己という名が結び付く抽象観念でもないだろう。前節の文脈を引き継いで「この\kenten{自己}」と表現されているところで、仮に自己が抽象観念だとすれば、前節で述べていたことは、複数の個々人に当てはまる自己という種の抽象観念が意識によってその同じ抽象観念になる次第を説明していることになるが、実際に前節で問題になっていたのはひとつの自己が同じ自己になる次第であり、抽象観念や自己一般が同じになる次第の説明ではなかったからである。こうして消去法的に、ここで「この\kenten{自己}」と呼ばれているものは、自己という種に分類されるところの任意の個物を示していることになるはずなのである。実際、これは直前の文脈でひとりの自己が同じ自己になる次第が説明されていたことと最もよく整合する。
\par
このように考えれば、「\kenten{人格}はこの\kenten{自己}に対する名である」という命題の解釈は次のようになる。すなわち、自己の抽象観念と合致する任意の個物(個人)は人格という名で呼ばれる、すなわち、9節で定義したような人格の抽象観念と合致するのである。しかし、この解釈にはまだ問題がある。その問題とは、この解釈を取った場合、人は自分にとっての自己以外に、自己と呼ぶことができる個物があると確実に知ることができないため、自分以外のいかなる人間をも人格であると見なすことができないということである。こう考える理由は次の通りである。まず、本稿2章1節と3章1節(前節)で述べたが、自己は意識によって構成されるものであり、しかも、人格に先立って第一に構成されるものなのであった。しかし2章5節で見たとおり、ロックはこの意識を私秘的な作用だと述べ、第三者はその当人の意識作用を知覚することができないと明言していた。そうだとすれば、人は自分以外に自己である個物があることを知ることができず、よって自分以外に人格があるということはできなくなるからである。
\par
しかしロックは、この難点を注意深い言葉遣いによって回避している。
\begin{quote}
どこであれ、ある人が、彼が\kenten{自分自身}({\itshape himself})と呼ぶものを見つけるところには、〔…〕同じ\kenten{人格}があると他の人はいうことができる(may say)。(2.27.26)
\end{quote}
ロックは、当人が自己自身と呼ぶ\kenten{もの}を第三者が人格と呼ぶとは言わない。当人にとっての自己を、第三者は知り得ないからだ。ロックが言っているのは、当人が自己自身と呼ぶものを\kenten{見つけるところ}に、第三者はその当人と同じ人格があると言うことができるということだけである。この微妙な言い回しに潜むロックの考えは次のようなものであろう。すなわち、第三者は、当人の意識作用(2章で述べた点から、記憶より専有と考えた方が良いと思われる)を想定した上で、その当人が自己と呼ぶだろうと蓋然的に判断(judgement)\footnote{ロックにおいて「判断」(judgement)とは、直観や論証によって絶対確実に知ることができない命題を、知られうるさまざまな手がかりを動員して蓋然的に認めること、あるいはそれによって認められた命題のことを意味する。ここではそうした蓋然的判断を、通常使われる意味での判断と区別し「蓋然的判断」という語で表す。 cf. Owen(2007), p. 406.}するものを、その人と同じ人格と呼ぶことができるのである。つまり、「人格」と呼ばれているのは当人にとっての自己なのではなく、その当人にとって自己であろうものとして発話者がその心の中で構成した観念なのである\footnote{ここまでの議論では、ある人のことを第三者が人格と呼ぶ場面のことだけを考えてきた。しかし、当然、当人が自身のことを人格だと呼ぶ場合も考えられる。この場合であれば、当人は当人にとって自己を直接意識によって知覚することができるので、当人についての絶対確実な人格を構成することができるだろう。しかし、実際に法廷用語として法廷で使われる人格概念はこうしたものではないだろう。こうした場合が人格概念の主な使われ方であるとは思わない。}。
\par
このことによって、件の命題を十全に解釈することができる。名とは、先ほど確認したように、\kenten{発話者の心の中}にある観念の記号なのであった。しかし、特定の自己の観念はその自己を意識している当人にしか得られない観念である。すると第三者が「人格」という名を用いる際、それが意味表示しているのは、\kenten{発話者が}当人のうちにあると蓋然的に判断した自己の観念なのである。すると、ある人が人格であると第三者が言うときには、次のような段階を踏んでいることになるはずだ。まず、発話者はある人の言動を観察することで、当人には意識があるだろうと蓋然的に判断する。次に発話者は、当人がその意識によって当人自身のことを自己であると通時的に認識するだろうと蓋然的に判断する。そして最後に、発話者は、発話者が持つ観念としての当人が、通時的に自己認識する知性的存在という観念に、すなわち人格の抽象観念に合致していることを見出すために、その当人を人格と呼ぶのである\footnote{この点は、チャーマーズのいう哲学的ゾンビ(zombie)を用いるとわかりやすい。哲学的ゾンビとは、ただ一点、意識経験(conscious experiences)が全くないこと以外すべてが人間と同じ生物である。それにもかかわらず、\kenten{人が}哲学的ゾンビを指して、その中に意識があると蓋然的に判断して「彼は人格である」と名付けることは正当なことであるだろう。 cf. Chalmers(1996), p. 94.}。私は、これが「\kenten{人格}はこの\kenten{自己}に対する名である」という命題の意味していることではないかと考える。
\par
そうすると、人格の同一性は意識の同一性に存するというロックの発言における「意識」は、特に法廷などの公共的な場面において、当人のうちにあるだろうと第三者によって蓋然的に判断されたものであることだろう。これによって、一ノ瀬が酔漢の事例と呼んでいた同一性章22節前半のロックの議論が理解できる。ロックは、ひとりの泥酔した男が「後になって決して意識しないとしても、酔った時に犯した事実のために罰される」のは、ひとえに「酔った男としらふの男は同じ人格」だからだと述べる。これは明らかに、私秘的な心的作用として考えられた同じ意識によって同じ人格になるという説明とは対立している。しかし、ロックはこれが「人間の法」(Humane Laws)の「その知り方に適した正義」(a Justice suitable to their way of Knowledge)であると述べるのだ。\footnote{ここで「人間の法」という表現をどう捉えるかだが、続く文脈が「やはり、人間の裁判官は正当に彼を罰するのである。〔…〕しかし、最後の審判の日には(in the great Day)〔…〕だれひとり、彼の何も知らないことについて責任を取らされるはずはなく、〔…〕」(2.27.22)と、神による最後の審判との対比であることから、ここで人間の法というのは、法を人間が実際に運用することとして解釈する。}そしてロックによれば、人間の裁判官は、被告が過去になした行為を自ら意識していないと証言したとしても、「実際のことか虚偽かを確実に区別することができない」。この場合、少なくとも同じ人間が悪事を働いたという「事実は彼に対抗して証明され、意識の欠如は彼のために証明されえないから」(2.27.22)「酔漢はもしかすると彼がしたことを意識していないかもしれないけれども、なお人間の裁判官は正当に彼を処罰する」(2.27.22)のである。ここで「人間の裁判官」が行っていることを整理しよう。彼が被告人の一人称的作用である意識について知らないのは確かだ。それにもかかわらず、彼は被告人を酔った時に為した行為のために正当に罰する。繰り返しになるが、このような罰が成り立つのは、酔った男としらふの男が同じ人格だからである。そうだとすれば、ここで裁判官は現在の被告人の人格と酔った男の人格の同一性を、彼の一人称的意識を知らないにもかかわらず、蓋然的な判断(judgement)を下しているのである。ロックによれば、人格の同一性は意識の同一性のみに存するので、明らかに裁判官がここでやっているのは意識の同一性の三人称的構成、そして同じことだが、人格同一性の三人称的構成である。
\par
ここでひとつ注意したいことがある。人格同一性の三人称的構成は、人格同一性を考える際に当人の一人称的同一性がまったく考慮に入れられる必要がないということを含意するのではない。意識の同一性が三人称的に構成される際には、「当人のなかに、一人称的に意識の同一性があるはずだ」という形式の想定であることが守られる必要があるはずだ。つまり、一人称的な意識の同一性は法廷における人格同一性を必ずしも直接には決定しないけれども、第三者による意識同一性の想定がいわば指針としてなくてはならないものなのである。実際、同一性章23節では、酔った男としらふの男が同じ人格だと判断されるのは、「意識の欠如が彼のために証明されえないから」であった。これに関して、もし、意識の欠如が証明されたならば、裁判官は彼を罰するわけにはいかなくなることになる。なぜなら、「人間の法は、〔…〕\kenten{正気の男}を\kenten{狂気の男}が為した行動のために罰しない」(2.27.20)からである。ここから、人格の三人称的構成には、その人が過去に為した出来事をその一人称的意識において正しく専有しているということが合理的に想定されることが必要だという条件が課せられていると考えられる。そして、このような条件を満たして構成された人格同一性は、現在の人格と過去の人格のあいだに専有という関係が想定され、それによって、過去の行為に対して現在処罰を与えることが可能になるのである。
\par
これが、私の提起する三人称的構成説の再解釈である。これを今後、\kenten{一人称的意識を想定した三人称的構成説}と述べることとしよう。これは、人格の一人称的側面を「自己」に、そして人格の三人称的側面をそれが「名」であることに求めることで、この相反するように見えるふたつの側面を調停しようとするものである。ここから、本章で解決することを目標にしていた、人格同一性論を一人称的構成説または三人称的構成説で捉えたときに立ち現れる問題点を解決できるかどうか考えてみることにする。まず、三人称的構成の問題点に対して、一人称的意識を想定した三人称的構成説がどのように応答できるかを考えてみることにしよう。
\par
三人称的構成説として考えたときの問題点は二つあった。ひとつめは、ロックが私秘的なものと語っていた意識を三人称的に構成することはどのようにして正当化されるのかという点である。これがなぜかといえば、名が発話者の心の中の観念を表すものだったからである。人格は自己に対する名だが、当人自身が発話者なのでない限り、発話者は自己であるものをそもそも知覚することができない。しかし、発話者は当人の言動を鑑みることで、当人が意識をもっており、それによって当人自身にとって自己であると蓋然的に判断する。人は他者の意識を知覚できないことと、それでも他者を人格と呼ぶことができるというこの間隙のうちに、意識の三人称的構成が正当化されるのである。こうした三人称的に構成された人格同一性の言明は蓋然的な判断であり、確実な知識ではない。そこで、このように反論されるかもしれない。すなわち、そうした不確実性を残す人格を帰責主体としてはいけないのではないか。これに対して、しかし、ロック自身は、確実な知識だけが人が用いる知識なのではないと考えている。なぜなら、ロックも強調するように、確実な知識だけでは人が対応しなければならない様々な出来事に対応しきれないからである(4.14.1)。人は実際に行為するときには確実な知識をえるまで探究をつづけるほど能天気であるわけにはいかず、それゆえ、蓋然的な判断に頼って行動するのである。そして人格と、つまり第三者と関わることは、「われわれにとって待ったなしのつねに差し迫った不可欠の」\footnote{一ノ瀬(1997), p. 169.}実践なのである。これらを勘案すれば、ロックが人間の社会的実践の場面で人格同一性が三人称的に、蓋然的判断によって構成されると考えたことはまったく自然な話だっただろうと思われる。かくして、人格が発話者の観念を表す「名」であることと、状況によっては蓋然的判断に甘んじることの正当性により、人格同一性は少なくとも法廷に代表される公共的な実践の場においては三人称的に構成されるものなのだ。
\par
三人称的構成説の次の難点は、実践的場面での探究・決定・同意という行為を通じて構成されることで人格が道徳的概念になるというモデルは、探究や決定などの行為を必要としないであろう知性をもつ神の前でどのようにして道徳的意味をもつようになるのかという問題である。この問題を、一人称的意識を想定した三人称的構成説は、当人と過去の当人の人格のうちに、専有する意識があったいうことを合理的に想定した上で三人称的に構成すると考えることによって解決する。一人称的意識による専有があったと想定することを媒介に、過去の人格と現在の人格が帰責主体として同一であることの権利が求められるのである。この道筋を取れば、人格概念は道徳的な場で探究、決定、同意という行為を通じて使用される事実に先立ってそれ自体で道徳的に使われる権利を有することになる。これによって、神の審判がどのようになされるにしろ、この人格が帰責主体として正当に使用されることのできる概念になるのである。
\par
最後に、一人称的構成説の難点に対して一人称的意識を想定した三人称的構成説がどのように応答できるかを考える。まず、バトラーの批判、すなわち、人格同一性を構成する意識自体のうちに人格同一性が前提されているという批判はあたらない。意識同一性についても三人称的に構成されるとする以上、人格同一性を構成するのは第三者の判断であり、そこには当人の人格同一性は先取されていないからである。これは先ほど注で、哲学的ゾンビに人格があるとすることができると述べたことからも明白であろう。次にリードの批判、すなわち、人格同一性が推移律を満たさないという批判に対しても応答ができる。この場合、専有説がリードの批判に対応したのと同じ仕方で応答することができる。すなわち、間接専有と直接専有を分け、将軍は少年のときの盗みを間接専有しているはずだと蓋然的に判断できた時に両者が同一人格であるということができるとすれば、推移律を満たした形で人格同一性を判断できる。
\par
最後に、錯誤の問題についても応答したい。あらかじめ述べておくと、一人称的意識を想定した三人称的構成説は、錯誤の問題を完全に解決することはできない。よって以下では、錯誤の問題に応答する際に、一人称的意識を想定した三人称的構成説ではどこまでのことが言えるかを描き出すこととする。結論から言うと、一人称的意識を想定した三人称的構成説は、当人の「私は過去のしかじかの行為を専有している」という一人称の心理的言明に対して、それが真であるか偽であるかを決定できる点において、単なる一人称的構成よりも進んだ解釈であるといえる。単なる一人称的構成説の場合、権威をもつのが当人だけであるので、当人が「私はしかじかの行為を専有している」と述べた場合、それは一人称の心理的言明であるので必ず真になってしまい、そのため、事実にかかわらず、ある行為を専有することによってのみその行為を為した人格となってしまう。これに対して、一人称的意識を想定した三人称的構成説では、「私は過去のしかじか行為を専有している」という当人の主張に対して、事実に照らしてそれが真であるか偽であるかを判断することができる。真なる専有とは、事実起こったことを自ら専有することであり、偽なる専有とは、事実としては起こらなかったことを妄想的に専有することである。これらを区別できる点で、一人称的意識を想定した三人称的構成説は単なる一人称的構成説よりも錯誤の問題に対する耐久性がある。
\par
ただ、一人称的意識を想定した三人称的構成説は、錯誤の問題について、ロックに沿った仕方で、これ以上確実なことをいうことはできない。もし、ここで偽である専有は人格同一性を構成できないとすることができたならば、一人称的意識を想定した三人称的構成説は錯誤の問題を解決することができる。だが他方、当人が専有していると主張している出来事が明らかに当人の妄想であったとしても、当人が実際その出来事を専有している判断するしかないような場合には、その行為が実際に起こったことでなくてもそれを尊重すべきという考え方もできる。事実としては起こらなかったことを実際に専有している時、人格同一性の構成においては、実際に専有していることと、専有しようとしている出来事が事実なかったことの、どちらをより重要視すべきかという点は、いまだ開かれたままなのである。専有が偽であった場合このような問題が起こることには原因がある。それは、ロック自身錯誤が起こる可能性を認めているにもかからわず、実際には「神の善性」(the Goodness of God)のおかげで錯誤が起こることはないということを(明らかに独断的に)認めて話を進めているからである(2.27.13)。ロックのこうした態度のために、一人称的な専有という意識作用とその事実に照らした真偽が食い違うときの問題については、ロック解釈の範疇では答えを出すことはできないのである。しかしながら、先に述べたところから、一人称的意識を想定した三人称的構成説は、単なる一人称的構成説よりも錯誤の問題を解決に近づけることができる解釈であるということは確かだと思われる。
\newpage
\chapter*{
結論と今後の展開
}\addcontentsline{toc}{chapter}{結論と今後の展開}
本論文では、「同じ意識」の解釈の仕方の諸説のうちに一人称的構成説と三人称的構成説という対立があることを示し、これらを調停する解釈を得ることを目標としてきた。結果、本論文は「人格は自己に対する名である」という命題から、意識の同一性、ひいては人格同一性は、それが法廷に代表される実践の場で使われるなかで問題にされる概念である限り、当人がその一人称的意識において過去の行為を専有しているはずだという三人称的想定によって判断されるものだと結論する。
\par
この際、人格同一性の一人称的権威は、三人称的判断が一人称的意識状態をめざすものであるということに従うべきだという規範の形でのみ考慮される。三人称的権威はこれに応じたならば、一人称的意識状態を想定すらしないで、客観的事実のみの考察によって直接判断することはできない。このように考えることで、ロックの人格同一性論を一人称的構成ないし三人称的構成よって捉えた際の難点をどれも解消するか、あるいは解消に近づけることができたと考えている。これが、本論文の結論である。
\par
最後に、この結論からどのように今後の研究を展開できるかを素描して本論文を終えようと思う。考えられる道筋は、この人格概念を通じてロックの政治哲学と認識論あるいは実践哲学の共通点を探し出していく道筋である。これは一ノ瀬\footnote{一ノ瀬(1997)}がすでに手掛けていることだが、彼が考える三人称的構成説と私の考える三人称的構成説とでは一人称的意識が人格であることに対して与える影響が異なるため、ロック解釈としても異なる結果を生み出すだろうと期待できる。

\chapter*{参考文献}
\addcontentsline{toc}{chapter}{参考文献}
\section*{一次文献}
\begin{itemize}
\item{} Locke, J. (1690). {\itshape An Essay concerning Human Understanding}, ed. Nidditch, P., Oxford University Press, 1975.
\begin{itemize}
\item{} 【邦訳】ロック, ジョン. (1690). 『人間知性論』(全4巻), 訳. 大槻春彦, 岩波書店, 1972-1977.
\end{itemize}
\end{itemize}
\section*{二次文献}
\begin{itemize}
\item{} Alston, W. \& Benett, J. (1988). “Locke on People and Substances”, in {\itshape The Philosophical Review}, vol. 97, no. 1, 1998, pp. 25-46.
\item{} Atherton, M. (1983). “Locke’s Theory of Personal Identity”, in {\itshape Midwest Studies in Philosophy VIII: Contemporary Perspectives on the History of Philosophy}, ed. Vere Chappell., University of Minnesota Press, 1983, pp. 273–93. 
\item{} Behan, D. P. (1979). “Locke on Persons and Personal Identity”,in {\itshape Canadian Journal of Philosophy}, vol. 9, no. 1, 1979, pp. 53-75.
\item{} Butler, J. (1736).  “Of Personal Identity”, in {\itshape The Analogy of Religion}, 1736, from ed. Perry, J. {\itshape Personal Identity}, California: University of California Press, 1975, pp. 99-105.
\item{} Chalmers, D. (1996). {\itshape The Conscious Mind: In Search of a Fundamental Theory}, Oxford: Oxford University Press, 1996.
\item{} Dicker, G. (2019) {\itshape Locke on Knowledge and Reality: A Commentary on an Essay Concerning Human Understanding}, Oxford: Oxford University Press, 2019.
\item{} 一ノ瀬, 正樹. (1997).『人格知識論の生成:ジョン・ロックの瞬間』東京大学出版会, 1997.
\item{} 今村, 健一郎. (2010).「ジョン・ロックの人格同一性論」, 『イギリス哲学研究』第
33号, 2010, pp. 19-33.
\item{} Kaufman, D. (2016). “Locke’s Theory of Identity”, in ed. Stuart, M. {\itshape A Companion to Locke}, Oxford: Blackwell, 2016, pp. 236-259.
\item{} Leibniz, G. W. (1765). {\itshape Nouveaux Essais sur l'entendement humain}, von {\itshape Sämtliche Schriften und Briefe}, Reihe. VI, Band. 6, Berlin: Preußische (jetzt Deutsche) Akademie der Wissenschaften, 1990.
begin{itemize}
\item[] 【邦訳】ライプニッツ, G. W. (1765) 『人間知性論』, 訳. 米山優., みすず書房, 1987.
end{itemize}
\item{} Mackie, J. L. (1976). {\itshape Problems from Locke}, Oxford: Clarendon Press, 1976.
\item{} Noonan, H(2003).  {\itshape Personal Identity},2nd. ed, Routledge, 2003.
\item{} Owen, D. (2007). “Locke on Judgement”, in ed. Newman.L, {\itshape The Cambridge Companion to Locke's “Essay concerning Human Understanding”}, Cambridge University Press, 2007, pp. 406-435.
\item{} Ried, T. (1785). “Of Mr. Locke’s Account of Our Personal Identity”, in {\itshape Essays on the Intellectual Powers of Man}, 1785,  from ed. Perry, J. {\itshape Personal Identity}, California: University of California Press, 1975, pp. 113-118.
\item{} Shoemaker, S. (1963). {\itshape Self-Knowledge and Self-Identity}, New York: Cornell University Press, 1963.
\item{} Yaffe, G. (2007). “Locke on Ideas of Identity and Diversity”, in ed. Newman, L. {\itshape The Cambridge Companion to Locke's “Essay concerning Human Understanding”}, Cambridge: Cambridge University Press, 2007, pp. 192-230.
\item{} Winkler, K. P. (1991). “Locke on Personal Identity”, in {\itshape Journal of the History of Philosophy}, vol. 29, no. 2, 1991, pp. 201-226.

\end{itemize}
\end{document}
